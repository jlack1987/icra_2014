\section{INTRODUCTION}

For as long as the field of robotics has existed, which has not been very long relative to other areas of engineering,
it has been a goal to make robots that look and behave like humans. Great strides have been made in the area of
robotics that have produced arms capable of all the motions of humans and with comparable strength. Similar advances
have been made with robotic hands. Hands such as the Sandia Hand \cite{url:Sandia_National_Labs} possess
incredible dexterity. Unfortunately, advancements in robotic walking have been much slower than much of
the rest of robotics; however, this is not due to a lack of effort. Bipedal robots have been walking for quite some time actually. Initially it
was shown that dynamic walking can be obtained without the use of control at all \cite{McGeer90a,CRTW05}.
The fact that passive approaches could be used to obtain walking points to the fact that there is a great amount
of energy stored in dynamic walking, and if harnessed correctly and paired with good control, dynamic walking can
be achieved with minimal input.

In addition to over actuation, there are phases of both full actuation and under actuation in human gaits. While full actuation is seen
everywhere in controls, under actuation is similar to over actuation in that it is not as common. Under actuation is,
however, an inherent phase of human walking and has been studied in the context of bipedal walking for some time. Under
actuation has been studied in great detail in point footed walking in \cite{GCS09,SGC07,WGCCM07,Ames12} to name just a few.

The vast majority of approaches for achieving bipedal robotic locomotion
are centered around controlling the Zero Moment Point(ZMP) \cite{FAA10,KKKFHYH06,TIOMA05}. Though ZMP walking has proven
successful on many robots, requiring the feet to be flat is extremely restrictive for walking gaits. In contrast, humanlike
walking consists of multiple domains consisting of both single and double support. There is a great collection of works that have studied the different domains of walking
as they pertain to bipedal locomotion \cite{SPSA11,SA09a}; however, all of the phases have yet to be put together collectively and
in a formal manner.

Bipedal walking robots are notorious for requiring large actuator torques at relatively low velocities. In many fields of robotics, this
can simply be remedied by the addition of increasingly larger actuators to compensate; however, due to the large number of
degrees of freedom coupled with the desire to design to anthropomorphic proportions, it is more critical than ever that
controllers be designed that are capable of producing behaviors that require as little torque as possible. This
can be done by producing walking behaviors that are highly dynamic and can make use of the passive
dynamics as humans do in order to reduce actuator requirements. It is also important for
robotics researchers to develop novel control algorithms capable of factoring in actuator requirements in
real time. This is important because regardless of how a behavior is designed, the behavior of the
controller outside of the designed region of operation is equally as vital since external disturbances
are inevitable.
\begin{figure}[!]
\centering
\includegraphics[width=80mm]{figures/AMBER2ROBOTS.pdf}
\caption{Pictures of the bipedal robotic testbed Amber 2.0.}
\label{fig:robots}
\end{figure}

This paper will first present an approach for achieving multidomain planar bipedal walking consisting of both
single and double support phases with the goal being to eventually realize the results presented on Amber 2.0. Amber
2.0 is a bipedal robot designed and built in AMBER Lab. A picture of Amber 2.0 as well as a SolidWorks model are
shown in \ref{fig:robots}. There are many works on planar bipedal walking both with feet \cite{CDG08}
and without feet \cite{SPPG10,YPA12} and performing behaviors from walking \cite{GCS2007} to running \cite{ZYA2012},
yet there has been virtually no works that have attempted to understand the multicontact phases of walking which
are a significant portion of the human gait. Furthurmore, a novel control approach for reducing actuator torques
via an online optimization will be shown to drastically reduce actuator requirements as well as remedy the
control issues inherent to double support.

The remainder of the paper will be structured as follows. Sect. II. will introduce the unpinned approach
used to model the bipedal robot as well as discuss the role external forcing. In Sect. III, we will
discuss the generalized control algorithm used to achieve multiphase walking before discussing the multidomain hybrid
system model in Sect. IV. Sections V and VI will discuss the human inspired aspect of our control approach by
discussing the human inspired output functions and a corresponding optimization for automatically generating
parameters that fully define our controllers. We will then present simulation results comparing the performance
of both controllers. 
